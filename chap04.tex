\chapter{Reference}
\label{cha:reference}


This chapter contains description of the function~\fun{dynopt}, the
main function of the collection of functions which extend the
capability of MATLAB Optimisation Toolbox, specifically of the
constrained nonlinear minimisation routine~\fun{fmincon}. The chapter
starts with section listing general descriptions of all the input and
output arguments and the parameters in the optimisation options
structure, continues with the function description, and ends with some 
tutorial. 

\section{Function Arguments}
\label{sec:funarg}

All input and output arguments to the \fun{dynopt} function are
described in this section. Section \ref{sec:inarg} describes all input
arguments built in input structure \argfun{optimparam}. Then output
arguments built in output structure \argfun{optimout} are treated in
section \ref{sec:outarg} and as last the optimisation options parameters
structure~\argfun{options} which is given by MATLAB is described in
Tab.~\ref{tab:optpar}. It is important to mention here, that the names
of input and output structures can be changed by user, but their fields
described later have to be used as described. 

\begin{table}[ht]
\begin{center}
\begin{tabular}{r p{10cm}}
\hline
$ni$ &-- number of intervals\\
$nx$ &-- number of state variables\\
$nu$ &-- number of control variables\\
$np$ &-- number of parameters\\
$nm$ &-- number of measurements\\
\hline
\end{tabular}
\caption[Predefined variables]{Some predefined variables which are used for
  function description}\label{tab:shortcuts}
\end{center}
\end{table}

Table \ref{tab:shortcuts} describes some predefined variables which
are used to simplify \fun{dynopt's} description in sections
\ref{sec:inarg} and \ref{sec:outarg}. 

\subsection{Input Arguments}
\label{sec:inarg}

As mentioned before, input arguments described bellow do entry
\fun{dynopt} in a structure called \argfun{optimparam}. This contains
them as fields, e.g., \argfun{optimparam.optvar}. \argfun{optimparam}
has following fields to be set: 
\begin{description}
\item[\argfun{optvar}] -- The choice of optimisation variables: 1 -
  times, 2 - control, 2 - parameters. Their combination is given by
  their summations, e.g., 3 - optimise times and control. All the
  possibilities are listed below
  \begin{enumerate}[1]
  \item - optimise times,
  \item - optimise control,
  \item - optimise times and control,
  \item - optimise parameters,
  \item - optimise times and parameters,
  \item - optimise control and parameters,
  \item - optimise all: times, control, and parameters.
  \end{enumerate}
\item[\argfun{objtype}] -- Parameter which defines the type of
  objective function to be minimised/maximised in optimisation. Two
  possible types of objective function may have been used: 
  \begin{description}
  \item[Mayer type] - if Mayer type objective function is used set the
    parameter \argfun{objtype} to an empty matrix.
  \item[Sum type] - if Sum type objective function is used, parameter
    \argfun{objtype} is a structure containing two variables
    \argfun{tm}, and \argfun{xm}. \argfun{tm} is a $nm$-by-1 vector of
    times, in which the measurements are taken. \argfun{xm} is a
    $nx$-by-$nm$ matrix of taken measurements in times
    \argfun{tm}. For more information about the types of objective
    functions see \fun{objfun} description in section
    \ref{sec:fundesarg}. 
  \end{description}
\item[\argfun{ncolx}] -- Parameter which represents the number of
  collocation points for state variables. This has always to be a
  number greater than zero. 
\item[\argfun{ncolu}] -- Parameter which represents the number of
  collocation points for control variables. It may have been defined
  as $[~]$ if control variable doesn't belong to optimisation
  variables and also doesn't occur in \fun{process}, \fun{objfun},
  \fun{confun}. Otherwise it has to be a number greater than zero.
\item[\argfun{li}] --  Parameter representing lengths of intervals. It
  has always to be filled with ni--by--1 vector of initial lengths of
  intervals. 
\item[\argfun{tf}] -- Parameter representing the final time, if the
  value of $t_{f}$ is not specified use empty brackets [~].
\item[\argfun{ui}] --  Parameter representing control variables
  applied on each time interval in \argfun{li}. As mentioned for
  \argfun{ncolu} parameter, if control variable is needed it has to be
  defined as nu--by--ni matrix of control variables for each
  interval. Otherwise it has to be an empty matrix [~]. 
\item[\argfun{par}] -- Parameter representing time independent
  parameters. As in \argfun{ui} also here it may have been defined
  either np--by--1 vector of time independent parameters or an empty
  matrix [~]. 
\item[\argfun{bdu}] -- Parameter representing bounds to the control
  variables. If not defined it has to be an empty matrix [~].  If
  control constraints are the same in each time interval then it has
  to be an nu--by--2 matrix: [$\ve{lbu}~\ve{ubu}$].  If control
  constraints are not the same in each time interval then it has to be
  an nu--by--ni*2 matrix: [$\ve{lbu}_1~\ve{ubu}_1, \ldots,
  \ve{lbu}_{n_i}~\ve{ubu}_{n_i}$] where $n_i$ is the number of intervals.
\item[\argfun{bdx}] -- Parameter representing bounds to the states.
  If not defined it has to be an empty matrix [~].  If state
  constraints are the same in each time interval then it has to be an
  nx--by--2 matrix: [$\ve{lbx}~\ve{ubx}$].  If state constraints are
  not the same in each time interval then it has to be an nx--by--ni*2
  matrix: [$\ve{lbx}_1~\ve{ubx}_1, \ldots,
  \ve{lbx}_{n_i}~\ve{ubx}_{n_i}$] where $n_i$ is the number of
  intervals.
\item[\argfun{bdp}] -- Parameter representing bounds to the
  parameters. If defined it has to be an np--by--2 matrix:
  [$\ve{lbp}~\ve{ubp}$], otherwise an empty matrix [~].
\item[\argfun{objfun}] --  The function to be
  optimised.~\subor{objfun} is the name of an M-file. For more
  information about this input argument, see section
  \ref{sec:fundesarg}. 
\item[\argfun{confun}] --  The function that computes the nonlinear
  equality and inequality constraints. \subor{confun} is the name of
  an M-file. For more information about this input argument, see
  section \ref{sec:fundesarg}.  
\item[\argfun{process}] -- The function that describes given
  process. \subor{process} is the name of an M-file. For more
  information about this input argument, see section
  \ref{sec:fundesarg}. 
\item[\argfun{options}] -- An optimisation options parameter structure
  that defines parameters used by the optimisation functions. This
  parameter is defined by MATLAB for all optimisation routines of
  MATLAB Optimization Toolbox. For information about the parameters 
  which are important for \fun{dynopt}, see Tab.~\ref{tab:optpar} or
  the individual function reference pages. These parameters can be
  specified using the function \fun{optimset}.

  In addition to these, parameter \argfun{NLPsolver} determines NLP
  solver using the \fun{fminsdp} toolbox. The default one is
  \fun{fmincon} but others can be specified as well if installed
  separately. This parameter can be set directly as
\begin{verbatim}
options.NLPsolver='ipopt';
\end{verbatim}
\end{description}

% continualna tabulka
%---------------------------------------------------------------------
\begin{longtable}{r||p{0.75\textwidth}}
\caption{Optimisation options parameters} \label{tab:optpar}\\
\hline \hline
{\textbf{Parameter Name}} & \multicolumn{1}{c}{\textbf{Description}}\\
\hline \hline
\endfirsthead
\caption*{concluded from previous page}\\
\hline
{\textbf{Parameter Name}} & \multicolumn{1}{c}{\textbf{Description}}\\ 
\hline \hline
\endhead
\multicolumn{2}{r}{{Continued on next page}}\\
\hline
\endfoot
\hline \hline
\endlastfoot
\hline
\argfun{DerivativeCheck} & Compare user-supplied analytic derivatives
 (gradients) to finite differencing derivatives (medium-scale algorithm
 only), default value: 'off'.\\  
\hline
\argfun{Diagnostics} & Print diagnostic information about the function to be
minimised or solved, default value: 'off'.\\ 
\hline
\argfun{DiffMaxChange} & Maximum change in variables for finite difference
derivatives (medium-scale algorithm only), default value: 0.1000.\\ 
\hline
\argfun{DiffMinChange} & Minimum change in variables for finite difference
derivatives (medium-scale algorithm only), default value:
1.0000e-008.\\ 
\hline
\argfun{Display} & Level of display. 'off' displays no output, 'iter' displays
output at each iteration, 'final' displays just the final output,
default value: 'final'.\\ 
\hline
\argfun{GradConstr} & Gradients for the nonlinear constraints defined by user,
default value: 'off'.\\ 
\hline
\argfun{GradObj} & Gradient for the objective function defined by user, default
value: 'off'.\\ 
\hline
LargeScale & User large-scale algorithm if possible, default value:
'on'.\\ 
\hline
\argfun{MaxFunEvals} & Maximum number of function evaluations allowed, default
value: '100*numberofvariables'.\\
\hline 
\argfun{MaxIter} & Maximum number of iterations allowed, default value: 400.\\
\hline 
\argfun{TolCon} & Termination Tolerance on the constraint violation, default
value: 1.0000e-006.\\
\hline 
\argfun{TolFun} & Termination Tolerance on the function value, default value:
1.0000e-006.\\ 
\hline
\argfun{TolX} & Termination Tolerance on x, default value: 1.0000e-006.\\
\hline
\argfun{TypicalX} & Typical x values (large-scale algorithm only), default
value: 'ones(numberofvariables,1)'.\\
\hline
\argfun{NLPsolver} & Choice of NLP solver, possible values:
'fmincon', 'snopt', 'ipopt', 'knitro', 'mma', 'gcmma', default: 
'fmincon'.\\
\hline
\end{longtable}
%---------------------------------------------------------------------

\subsection{Output Arguments}
\label{sec:outarg}

As for input arguments, the same holds for output arguments. That
means that the output arguments described bellow do leave \fun{dynopt}
in a structure called \argfun{optimout}. This contains them as fields,
e.g., \argfun{optimout.nlpx}. \argfun{optimout} has following fields: 
\begin{description}
\item[\argfun{nlpx}] -- holds the solution found by the
  \fun{dynopt}. If \argfun{exitflag} $>$ 0, then \argfun{nlpx} is a
  solution  otherwise, \argfun{nlpx} is the value the optimisation
  routine was at when it terminated prematurely. Vector \argfun{nlpx}
  contains all the parameters $\Delta \zeta_{i}, \ve{u_{ij}},
  \ve{x_{ij}}, \ve{p}$ defined in the NLP formulation in section
  \ref{sec:nlp-fp}. 
\item[\argfun{fval}] -- holds the value of the objective
  function in~\argfun{objfun} at the solution~\argfun{nlpx}.
\item[\argfun{exitflag}] -- represents the exit condition of
  optimisation. \argfun{exitflag} may be:
  \begin{description}
  \item[$>0$] indicates that the function converged to a solution
    \argfun{nlpx}, 
  \item[$0$] indicates that the maximum number of function evaluations
    or iterations was reached, 
  \item[$<0$] indicates that the function did not converge to a
    solution. 
  \end{description}
\item[\argfun{output}] -- represents an output structure that contains
  information about the results of the
  optimisation. \argfun{output.iterations} gives the information about
  the number of iteration, \argfun{output.funcCount} gives the
  information about the number of function  evaluations,
  \argfun{output.algorithm} returns the used algorithm,
  \argfun{output.stepsize} returns the taken final stepsize
  (medium-scale algorithm only), \argfun{output.firstorderopt} gives
  the information about a measure of first-order optimality
  (large-scale algorithm only). 
\item[\argfun{lambda}] -- The Lagrange multipliers at the solution
  \argfun{nlpx}. \argfun{lambda} is a structure where each field is
  for a different constraint type. \argfun{lamdba.lower} for the lower
  bounds lb, \argfun{lambda.upper} for the upper bounds ub,
  \argfun{lambda.ineqlin} for the linear inequalities,
  \argfun{lambda.eqlin} for the linear equalities,
  \argfun{lambda.ineqnonlin} for the nonlinear inequalities,
  \argfun{lambda.eqnonlin} for the nonlinear equalities.
\item[\argfun{grad}] -- holds the value of the gradient of objfun at
  the solution \argfun{nlpx}.   
\item[\argfun{t}] -- is a vector of times for optimal control profile
  returned by \fun{dynopt}.
\item[\argfun{u}] -- is a vector/matrix of optimal control profiles
  returned by \fun{dynopt}.
\item[\argfun{p}] -- is a vector/empty matrix of the optimal values of
  the parameters. 
\end{description}

Function parameters described in section \ref{sec:outarg}, and
Tab.~\ref{tab:optpar} are implicitly given by MATLAB Optimization
Toolbox for all it's subroutines. They also present parameters useful
for~\fun{dynopt} through function~\fun{fmincon}. 

\section{Function Description}
\label{sec:fundes}

\subsection{Purpose} 
\label{sec:fundespur}

The actual version of~\fun{dynopt} is able to solve dynamic
optimisation problems which cost functions can be expressed either in
the Mayer form or in the Sum form. The problem formulation can be
described by following set of DAEs: 
\begin{equation}
\min_{\ve{u}(t),\ve{p}}\mf{G}(\ve{x}(t_{f}),t_{f},\ve{p})
\label{eq:objmajer}  
\end{equation} or
\begin{equation}
\min_{\ve{u}(t),\ve{p}} \sum_{i=1}^{nm} 
\mf{S}(t_{i},\ve{x}(t_{i}),\ve{u}(t_{i}),\ve{p},\ve{x}^{\text{mes}}(t_{i}))
\label{eq:objsum}  
\end{equation} 
such that
\begin{align*}
\ve{M\dot{x}}(t)&=\ve{f}(t,\ve{x}(t),\ve{u}(t),\ve{p})\\ 
\ve{x}(t_{0})&=\ve{x}_{0}(\ve{p})\\ 
\ve{h}(t,\ve{x},\ve{u},\ve{p}) &= \ve{0}\\
\ve{g}(t,\ve{x},\ve{u},\ve{p}) &\leq \ve{0}\\
\ve{x}^{L}_i\leq &\ve{x}(t)\leq \ve{x}^{U}_i, \quad i=1,\ldots, n_i\\ 
\ve{u}^{L}_i\leq &\ve{u}(t)\leq \ve{u}^{U}_i, \quad i=1,\ldots, n_i\\
\ve{p}^{L}\leq &\ve{p}\leq \ve{p}^{U}
\end{align*}
with the following nomenclature:
\begin{description}
\item[$\mf{G}(\cdot)$] -- objective function in Mayer form evaluated
  at final conditions,  
\item[$\sum_{i=1}^{nm} \mf{S}(\cdot)$]  -- objective function of Sum
  form evaluated in times of taking the measurements $t_{i}$, 
\item [$\ve{M}$] -- a constant mass matrix,
\item [$\ve{h}$] -- equality design constraint vector,
\item [$\ve{g}$] -- inequality design constraint vector,
\item [$\ve{x}(t)$] -- state profile vector,
\item [$\ve{u}(t)$] -- control profile vector,
\item [$\ve{p}$] -- vector of time independent parameters,
\item [$\ve{x}_0$] -- initial conditions for state vector,
\item [$\ve{x}^{L}_i,\ve{x}^{U}_i$] -- state profile bounds on
  each interval $i=1,\ldots, n_i$, 
\item [$\ve{u}^{L}_i,\ve{u}^{U}_i$] -- control profile bounds on
  each interval $i=1,\ldots, n_i$, 
\item [$\ve{p}^{L},\ve{p}^{U}$] -- bounds to the parameters.
\end{description} 

\subsection{Syntax and Description}
\label{sec:fundessyndes}
 
\begin{verbatim} 
[optimout,optimparam]=dynopt(optimparam)
\end{verbatim} 
starts with the initial lengths of intervals \argfun{li}, initial
control values for each interval \argfun{ui} for defined number of
collocation points for state variables \argfun{ncolx}, and for control
variables \argfun{ncolu} to the final time \argfun{tf}, and minimises
either a Mayer type \argfun{objfun} evaluated in the final time or Sum type
\argfun{objfun} subject to the nonlinear inequalities or equalities
defined in \argfun{confun} for time $t_0$, $t_f$ or over full time
interval characterised by flag in \fun{confun} subject to a given
system in \argfun{process} with the optimisation parameters specified
in the structure \argfun{options}, with the defined set of lower and
upper bounds on the control variables \argfun{bdu}, state variables
\argfun{bdx}, and time independent parameters \argfun{bdp} so that
solution is always in the range of these bounds. All before mentioned
variables do entry \fun{dynopt} in \argfun{optimparam} structure. The
solution is returned in the \argfun{otpimout} structure described in
section \ref{sec:outarg}.   

\subsection{Arguments}
\label{sec:fundesarg}

The arguments passed into the function are described in section
\ref{sec:inarg}. The arguments returned by the function are described
in section \ref{sec:outarg}. Details relevant to \fun{dynopt} are
included below for \argfun{objfun}, \argfun{confun}, \argfun{process}. 

\begin{description}
\item[\argfun{objfun}] The function to be minimised. \argfun{objfun}
  is a string containing the name of an M-file function, e.g.,
  \subor{objfun.m}. Whereas \fun{dynopt} optimises a given performance
  index 
  \begin{description}
  \item[Mayer form \eqref{eq:objmajer}] objective function is
    evaluated at the final time $t_{f}$, thus \fun{objfun} takes a
    scalar \argfun{t}~-~final time $t_{f}$, scalar/vector
    \argfun{x}~-~the state variable(s), scalar/vector \argfun{u}~-~the
    control variable(s), both evaluated at corresponding  final time
    $t_{f}$, scalar/vector \argfun{p}~-~time independent parameters,
    and returns a scalar value \argfun{f} of the objective function
    evaluated at these value. The M-file function has to have the
    following form: 
    {\small \verbatiminput{userfunctions/majerobjfun.m}} 
  \item[Sum form \eqref{eq:objsum}] objective function is evaluated in
    the times of taking measurements $t_{i}$, thus \fun{objfun} takes
    a scalar \argfun{t}~-~time of taking measurements $t_{i}$,
    scalar/vector \argfun{x}~-~state variable(s), \argfun{u}~-~the
    control variable(s), both evaluated at corresponding time $t_{i}$,
    scalar/vector \argfun{p}~-~time independent parameters,
    scalar/vector \argfun{xm}~-~measured variable(s) in the above
    mentioned time $t_{i}$, and returns a scalar value \argfun{f} of
    the objective function evaluated at these values. The M-file
    function has to have the following form:   
    {\small \verbatiminput{userfunctions/sumobjfun.m}} 
   \end{description}
   If the gradients of the objective function can also be computed
   and \argfun{options.GradObj} is 'on', as set by options =
   optimset('GradObj','on') then the function~\fun{objfun} must return,
   in the second output argument, the structure \argfun{Df} holding the
   gradient values with respect to time \argfun{t}, states \argfun{x},
   controls \argfun{u} and parameters \argfun{p} as follows:
   {\small \verbatiminput{userfunctions/objfungrad.m}}
   The gradients \argfun{Df.t}, \argfun{Df.x}, \argfun{Df.u},
   \argfun{Df.p} are the partial derivatives of \argfun{f} at the
   points \argfun{t}, \argfun{x}, \argfun{u}, \argfun{p}. That means,
   \argfun{Df.t} is the partial derivative of \argfun{f} with respect
   to the \argfun{t}, the ith component of \argfun{Df.x} is the partial
   derivative of \argfun{f} with respect to the ith component of
   \argfun{x}, the ith component of \argfun{Df.u} is the partial
   derivative of \argfun{f} with respect to the ith component of
   \argfun{u}, the ith component of \argfun{Df.p} is the partial
   derivative of \argfun{f} with respect to the ith component of
   \argfun{p}.  
\item[\argfun{confun}] The function that computes the nonlinear
  inequality constraints $\ve{g}(t,x,u,p)<=\ve{0}$ marked as output
  argument \argfun{c} and nonlinear equality constraints
  $\ve{h}(t,x,u,p)=0$, marked as output argument \argfun{ceq}. As
  mentioned before, \fun{dynopt} optimises a given performance index
  subject to the constraints defined in corresponding flag:
  \begin{description}
  \item[\argfun{flag} = 0] the constraints are implied at the
    beginning $t = t_{0}$,
  \item[\argfun{flag} = 1] the constraints are implied over the whole
    time interval $t \in [t_{0},t_{f}]$,
  \item[\argfun{flag} = 2] the constraints are implied at the end $t
  = t_{f}$.
  \end{description}
  \argfun{confun} is a string containing the name of an M-file
  function, e.g., \subor{confun.m}. \fun{confun} takes a
  scalar~\argfun{t}~-~time value corresponding to the time $t$,
  scalar/vec\-tor \argfun{x}~-~state variable value(s), and
  scalar/vector \argfun{u}~-~control variable value(s) both
  corresponding to the value of \argfun{t}, scalar/vector
  \argfun{p}~-~time independent parameters, and returns two arguments,
  a vector \argfun{c} of the nonlinear inequalities and a vector
  \argfun{ceq} of the nonlinear equalities, both evaluated at
  \argfun{t}, \argfun{x}, \argfun{u}, \argfun{p} for given flag. For
  example, if confun=@confun, then the M-file \subor{confun.m} would
  have the form: 
  {\small \verbatiminput{userfunctions/confun.m}}
  If the gradients of the constraints can also be computed and the
  \argfun{options.GradConstr} is 'on', as set by \argfun{options} =
  optimset('GradConstr','on') then \argfun{confun} is a string
  containing the name of an M-file function, e.g.,
  \subor{confun.m}. The function \fun{confun} must return, in the
  third and fourth output argument, structures \argfun{Dc}, and
  \argfun{Dceq} holding the gradient values \argfun{t}, \argfun{x},
  \argfun{u}, \argfun{p} with respect to themselves. 
  {\small \verbatiminput{userfunctions/confungrad.m}}
  The gradients \argfun{Dc.t}, \argfun{Dc.x}, \argfun{Dc.u},
  \argfun{Dc.p} are the partial derivatives of \argfun{c} at the
  points \argfun{t}, \argfun{x}, \argfun{u}, \argfun{p}. That means,
  \argfun{Dc.t} is the partial derivative of \argfun{c} with respect
  to \argfun{t}, the ith component of \argfun{Dc.x} is the partial
  derivative of \argfun{c} with respect to the ith component of
  \argfun{x}, the ith component of \argfun{Dc.u} is the partial
  derivative of \argfun{c} with respect to the ith component of
  \argfun{u}, the ith component of \argfun{Dc.p} is the partial
  derivative of \argfun{c} with respect to the ith component of
  \argfun{p}, and the gradients \argfun{Dceq.t}, \argfun{Dceq.x},
  \argfun{Dceq.u}, \argfun{Dceq.p} are the partial derivatives of
  \argfun{ceq} at the points \argfun{t}, \argfun{x}, \argfun{u},
  \argfun{p}. 
\item[\argfun{process}] The function which describes process model,
  that means the right hand sizes of ODE or DAE equations. If the
  process model is described by system of ODE's the mass matrix M in
  \argfun{flag} = 7 shall be left an empty matrix, because of being
  set by \fun{dynopt} by default. If the system is describe by DAE's
  the mass matrix M in \argfun{flag} = 7 should be singular.
  \argfun{process} is a string containing the name of an M-file
  function, e.g., \subor{process.m}. \fun{process} takes a time
  \argfun{t}, scalar/vector of state variable \argfun{x}, scalar
  \argfun{flag}, scalar/vector of control variable \argfun{u}, both
  corresponding to time \argfun{t}, and scalar/vector of time
  independent parameters \argfun{p}, and returns \argfun{sys} values
  with respect to \argfun{flag} value evaluated at time \argfun{t}.
  The M-file function has to be written in the following form: 
  {\small \verbatiminput{userfunctions/process.m}}
\end{description}


\subsection{Algorithm}

\begin{description}
\item[Large-scale optimisation] By default~\fun{dynopt} will choose
  the large-scale algorithm of~\fun{fmincon} if the user supplies the
  gradient in objfun (and~\argfun{GradObj} is 'on'
  in~\argfun{options}) and if only upper and lower bounds exists or
  only linear equality constraints exist. This algorithm is a subspace
  trust region method and is based on the interior-reflective Newton
  method described in ~\cite{col96}. Each iteration involves the
  approximate solution of a large linear system using the method of
  preconditioned conjugate gradients (PCG). See the trust-region and
  preconditioned conjugate gradient method descriptions in the
  Large-Scale Algorithms chapter in~\cite{col99}.
\item[Medium-scale optimisation] \fun{dynopt} uses through
  the~\fun{fmincon} Sequential Programming (SQP) method. In this
  method, a Quadratic Programming (QP) subproblem is solved at each
  iteration. An estimate of the Hessian of the Lagrangian is updated
  at each iteration using the BFGS formula~\cite{col96}.

  A line search is performed using a merit function similar to that
  proposed by~\cite{han77}. The QP subproblem is solved using an
  active set strategy similar to that described in~\cite{gil81}.
  A full description of this algorithm is found in the Constrained
  optimisation section of the Introduction to algorithms chapter of
  the Optimization Toolbox manual. See also the SQP implementation
  section in the Introduction to Algorithms chapter for more details
  on the algorithm used.
\end{description}

The software layer of \fun{fminsdp} toolbox adds different NLP solvers
that have to be installed separately. This version of \fun{dynopt} was
tested with open source solver Ipopt 3.12.4. Note that ipopt requires
user supplied gradients in the cost functions and constraints. If
these are not supplied, fmincon is used instead. For concrete options
of this solver and other available other ones see their
documentations.

\section{Additional Functions}
\label{sec:addfun}

In this section, two functions are presented: \fun{profiles}, which
prepares plot-able state and control profiles and \fun{constraints},
which prepares a user given equality and inequality plot-able
constraints from the optimisation results returned in
\argfun{optimout}.

\subsection{Function~\fun{profiles}}
\label{sec:profiles}

\begin{verbatim}
[tplot,uplot,xplot] = profiles(optimout,optimparam,ntimes)
\end{verbatim}
takes an optimal output \argfun{optimout} and other input arguments
\argfun{optimparam} described in section \ref{sec:inarg}, and returns
vector \argfun{tplot}, vector/matrix \argfun{uplot}, vector/matrix
\argfun{xplot} with respect to \argfun{ntimes} which defines the
number of points plotted per interval.

\subsection{Function~\fun{constraints}}
\label{sec:constraints}

\begin{verbatim}
[tp,cp,ceqp] = constraints(optimout,optimparam,ntimes)
\end{verbatim}
takes an optimal output \argfun{optimout} returned by \fun{dynopt}, and
other input arguments \argfun{optimparam} described in section
\ref{sec:inarg}, and returns vector \argfun{tp},nonlinear inequality 
constraint vector/matrix \argfun{cp}, nonlinear equality constraint
vector/matrix \argfun{ceqp} defined in \argfun{confun} with respect to
\argfun{ntimes} which defines the number of points plotted per interval.

It is simple to make a graphical representation of obtained results by
using MATLAB's \fun{plot} function.



%%% Local Variables: 
%%% mode: latex
%%% TeX-master: "dynopt_guide"
%%% End: 
