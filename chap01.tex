\chapter{Before You Begin}
\label{cha:beforebegin}

\section{What is~\fun{dynopt}}
\label{sec:whatisdynopt}

\fun{dynopt} is a set of MATLAB functions for determination of optimal
control trajectory by given description of the process, the cost to be
minimised, subject to equality and inequality constraints, using
orthogonal collocation on finite elements method. 

The actual optimal control problem is solved by complete
parametrisation both the control and the state profile vector. That
is, the original continuous control and state profiles are
approximated by a sequence of linear combinations of some basis
functions. It is assumed that the basis functions are known and
optimised are the coefficients of their linear combinations. In
addition, each segment of the control sequence is defined on a time
interval whose length itself may also be subject to
optimisation. Finally, a set of time independent parameters may
influence the process model and can also be optimised.

It is assumed, that the optimised dynamic model may be described by a
set of ordinary differential equations (ODE's) or
differential-algebraic equations (DAE's).

This collection of functions extend the capability of the MATLAB
Optimization Toolbox, specifically of the constrained nonlinear
minimisation routine~\fun{fmincon}. 


\section{What is New in this Version}
\label{sec:what-new-this}


Version 4 introduces several new properties of the package:
\begin{itemize}
\item three type of constraints can be defined in the same
  time:~constraints in $t_{0}$, constraints over full time interval 
  $[t_{0},t_{f}]$, and constraints in $t_{f}$. Previously only one of
  them was possible.
\item time independent parameters are introduced into the \fun{process}
  function, \fun{objfun} function and \fun{confun}.
\end{itemize}

\section{How to Use this Manual}
\label{sec:howtouseman}

This manual has four main parts:
\begin{description}
\item[Chapter 2] introduces implementation of orthogonal collocation
  on finite elements method into general optimisation problems.
\item[Chapter 3] provides a tutorial for solving different
  optimisation problems.
\item[Chapter 4] provides a detailed reference description
  of~\fun{dynopt} function. Reference descriptions include the
  function syntax, detailed information about arguments to the
  function, including relevant optimisation options parameters.
\item[Chapter 5] provides some more examples solved by~\fun{dynopt},
  their definitions and solutions.
\end{description}

\section{Installing~\fun{dynopt}}
\label{sec:instdynopt}

\fun{dynopt} code does not need any special installation procedure. To
use~\fun{dynopt}, just add the Dynamic Optimisation Tool directory
\subor{dynoptim} into the path by addpath environment. 

As mentioned before,~\fun{dynopt} is a set of functions that extend
the capability of the MATLAB Optimization Toolbox. That means, that
for using~\fun{dynopt} this toolbox has to be provided. To determine
if the Optimization Toolbox is installed on your system, type this
command at the MATLAB prompt:
\begin{verbatim}
ver
\end{verbatim}
After entering this command, MATLAB displays information about the
version of MATLAB you are running, including a list of all toolboxes
installed on your system and their version numbers.  If the
Optimization Toolbox is not installed, check the Installation Guide
for instructions on how to install it.

\fun{dynopt} has been developed and tested since MATLAB 6.5 (R13). The
results in this guide are obtained with MATLAB 2013b (Linux 64b) using
SQP or interior point solver. It is quite usual that results obtained
and convergence criteria achieved with different versions of MATLAB or
its toolboxes can produce slightly different (better, worse) results.



%%% Local Variables: 
%%% mode: latex
%%% TeX-master: "dynopt_guide"
%%% End: 
