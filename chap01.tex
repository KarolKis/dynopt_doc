\chapter{Before You Begin}
\label{cha:beforebegin}

\section{What is~\fun{dynopt}}
\label{sec:whatisdynopt}

\fun{dynopt} is a set of MATLAB functions for determination of optimal
control trajectory by given description of the process, the cost to be
minimised, subject to equality and inequality constraints, using
orthogonal collocation on finite elements method. 

The actual optimal control problem is solved by complete
parametrisation both the control and the state profile vector. That
is, the original continuous control and state profiles are
approximated by a sequence of linear combinations of some basis
functions. It is assumed that the basis functions are known and
optimised are the coefficients of their linear combinations. In
addition, each segment of the control sequence is defined on a time
interval whose length itself may also be subject to
optimisation. Finally, a set of time independent parameters may
influence the process model and can also be optimised.

It is assumed, that the optimised dynamic model may be described by a
set of ordinary differential equations (ODE's) or
differential-algebraic equations (DAE's).

\section{What is New}
\label{sec:what-new-this}

\subsection*{Version 5}
Version 5 does not bring any improvements in the algorithm but focuses
on ease of implementation. In the default configuration, it uses
automatic differentiation package Adigator~\cite{wei17} to construct
Jacobians of process differential equations, cost function, and
constraints. This has a consequence that the problem source files from
version 4 or earlier are not compatible and have to be modified. On
the other hand, it eases the implementation considerably and reduces
errors in problem definition. However, manual specification of
Jacobians is still possible if needed.

\subsubsection*{Incompatibility with Version 4}
When files from older releases are to be reused, please keep in mind
following items:
\begin{itemize}
\item Main file does not need to turn on gradients in NLP options.
\item Files \subor{process.m}, \subor{objfun.m}, \subor{confun.m} must
  not contain \fun{case} statement -- use \fun{if/else} instead. 
\item File \subor{process.m} contains only information for \fun{flag}
  equal to 0 (differential euqations) and 5 (initial conditions) -- no
  gradients. Mass matrix information was moved to main file.
\item File \subor{objfun.m} does not contain information about gradients.
\item File \subor{confun.m} does not contain information about gradients.
\item Gradient information can be moved to newly defined files.
\end{itemize}

\subsection*{Version 4}
Version 4 introduces several new properties of the package:
\begin{itemize}
\item three type of constraints can be defined in the same
  time:~constraints in $t_{0}$, constraints over full time interval 
  $[t_{0},t_{f}]$, and constraints in $t_{f}$. Previously only one of
  them was possible.
\item time independent parameters are introduced into the \fun{process}
  function, \fun{objfun} function and \fun{confun}.
\end{itemize}
Version 4.2 enables definition of min/max constraints on state and
control independently on each interval (see
Section~\ref{sec:unconprobgradbound}).

Version 4.3 makes possible to use different NLP solvers (see
Chapter~\ref{sec:conprobgrad} for example of use Ipopt and
Chapter~\ref{cha:reference} for reference).

\section{How to Use this Manual}
\label{sec:howtouseman}

This manual has four main parts:
\begin{description}
\item[Chapter 2] introduces implementation of orthogonal collocation
  on finite elements method into general optimisation problems.
\item[Chapter 3] provides a tutorial for solving different
  optimisation problems.
\item[Chapter 4] provides a detailed reference description
  of~\fun{dynopt} function. Reference descriptions include the
  function syntax, detailed information about arguments to the
  function, including relevant optimisation options parameters.
\item[Chapter 5] provides some more examples solved by~\fun{dynopt},
  their definitions and solutions.
\end{description}

\section{Code, Documentation, and License}
Starting with the version 4.3, \fun{dynopt} has been moved to
bitbucket and both source code of the toolbox and this documentation
are available. 

Project: \\
\url{https://bitbucket.org/dynopt/}

Code:\\
\url{https://bitbucket.org/dynopt/dynopt_code}

Documentation:\\
\url{https://bitbucket.org/dynopt/dynopt_doc}

\medskip
\noindent
dynopt is free of charge to use and is openly distributed, but note
that (YALMIP license)
\begin{itemize}
\item Copyright is owned by Miroslav Fikar.
\item dynopt must be referenced when used in a published work (give me
  some credit for saving your valuable time!)
\item dynopt, or forks or versions of dynopt, may not be
  re-distributed as a part of a commercial product unless agreed upon
  with the copyright owner (if you make money from dynopt, let me in
  first!)
\item dynopt is distributed in the hope that it will be useful, but
  WITHOUT ANY WARRANTY; without even the implied warranty of
  MERCHANTABILITY or FITNESS FOR A PARTICULAR PURPOSE (if your
  satellite crash or you fail your PhD due to a bug in dynopt, your
  loss!).
\item Forks or versions of dynopt must include, and follow, this
  license in any distribution.
\end{itemize}
 
\section{Installation}
\label{sec:instdynopt}

\subsection{Manual Installation}
Download \fun{dynopt} from:\\
\url{https://bitbucket.org/dynopt/dynopt_code/downloads/dynopt_5.0.0.zip}. 

\fun{dynopt} needs \fun{fminsdp} toolbox~\citep{fminsdp} and
\fun{adigator} toolbox~\citep{wei17} to be installed. For convenience,
both are packed within the zip file, but feel free to get them
from their original locations. To install them, follow directions to add
respective directories to MATLAB path.

Next, install \fun{dynopt}: add the directory \subor{dynoptim} to
MATLAB path. 

All packages can be installed by getting into \subor{dynopt}
directory in MATLAB and running the following commands:
\begin{verbatim}
currpath = pwd;
addpath([currpath,filesep,'dynoptim']);
addpath([currpath,filesep,'adigator']);
addpath([currpath,filesep,'adigator',filesep,'lib']);
addpath([currpath,filesep,'adigator',filesep,'lib',filesep,'cadaUtils']);
addpath([currpath,filesep,'adigator',filesep,'util']);
addpath([currpath,filesep,'fminsdp']);
addpath([currpath,filesep,'fminsdp',filesep,'interfaces']);
addpath([currpath,filesep,'fminsdp',filesep,'utilities']);
\end{verbatim}
and then saving the path for future.

\subsection{Installation using tbxmanager}
\fun{tbxmanager} (\url{https://www.tbxmanager.com}) is a tool to
install and manage several third party optimisation toolboxes for
MATLAB.  The complete list of available packages is disponible at its
home page.

Install \fun{tbxmanager} according to directions at its home
page. Then, \fun{dynopt} is installed (or upgraded) as:
\begin{verbatim}
tbxmanager install dynopt
tbxmanager update dynopt
\end{verbatim}

\subsection{Optimisation Toolbox}
\fun{dynopt} is a set of functions that extend the capability of the
MATLAB Optimization Toolbox. That means, that for using~\fun{dynopt}
this toolbox has to be provided. To determine if the Optimization
Toolbox is installed on your system, type this command at the MATLAB
prompt:
\begin{verbatim}
ver
\end{verbatim}
After entering this command, MATLAB displays information about the
version of MATLAB you are running, including a list of all toolboxes
installed on your system and their version numbers.  If the
Optimization Toolbox is not installed, check the Installation Guide
for instructions on how to install it.

\fun{dynopt} has been developed and tested since MATLAB 6.5 (R13). The
results in this guide are obtained with MATLAB 2020a (Linux 64b) using
solvers \fun{fmincon} and \fun{ipopt}. Please, note that it is quite
usual that results obtained and convergence criteria achieved with
different versions of MATLAB or its toolboxes can/will produce
slightly different (better, worse) results.


%%% Local Variables: 
%%% mode: latex
%%% TeX-master: "dynopt_guide"
%%% End: 
